\cleardoublepage
\thispagestyle{empty}

\begin{center}
{\LARGE\bfseries\titulo: \subtitulo}\\
\end{center}
\begin{center}
\autor
\end{center}

\bigskip
\noindent{\textbf{Palabras clave}: \textit{\keywords}\\

\section*{Resumen}
Este documento expone mi trabajo de fin de grado, y los contenidos asociados al mismo.\\

Este trabajo tiene como objetivo explorar las diferentes posibilidades que puede aportar la realidad aumentada para dispositivos móviles aplicada a juegos de tablero. Para ello el proyecto se va a centrar en la planificación y desarrollo de un juego de mesa, que mediante las tecnologías de realidad aumentada, mas concretamente ARCore, aportará un nuevo enfoque sobre los juegos de esta temática, aprovechando las singulares características que la realidad aumentada ofrece.\\

El proyecto explorará también la integración de diferentes formas de interacción en juegos, como pueden ser las interfaces tangibles, por tanto, se crearán funcionalidades que se llevarán a cabo mediante interacción con elementos físicos, y otras funcionalidades que se desarrollarán con interacción únicamente con el dispositivo móvil.\\

El proceso de planificación y desarrollo del proyecto se llevará a cabo mediante el uso de metodologías ágiles y diseño centrado en el usuario.

\cleardoublepage
\thispagestyle{empty}

\begin{center}
{\LARGE\bfseries\tituloEng: \subtituloEng}\\
\end{center}
\begin{center}
\autor
\end{center}

\bigskip
\noindent{\textbf{Keywords}: \textit{\keywordsEng}\\

\section*{Abstract}
This document shows my end-of-degree project and the contents associated with it.\\

The purpose of this project is to explore the different possibilities that augmented reality for mobile devices can bring applied to board games. For this, the project will focus on the planning and development of a board game, that using the augmented reality technologies available, specifically using ARCore, will bring a new approach to the games of this theme, taking advantage of the singular characteristics that augmented reality offers.\\

The project will also explore the integration of different ways of interaction in games, allowing functionalities that will have to be done by interacting with physical elements, and other functionalities that will have to be done interacting only with the mobile device.\\

The project planning and development process will be carried out through the use of agile methodologies and user-centered design.

\chapter*{}
\thispagestyle{empty}

\noindent\rule[-1ex]{\textwidth}{2pt}\\[4.5ex]

Yo, \textbf{\autor}, alumno de la titulación \textbf{\grado} de la \textbf{\escuela} de la \textbf{\universidad}, con DNI 71358141C, autorizo la ubicación de la siguiente copia de mi Trabajo de Fin de Grado en la biblioteca del centro para que pueda ser consultada por las personas que lo deseen.\\

Así mismo, el código fuente del proyecto y esta documentación pueden consultarse en la dirección \url{https://github.com/matl1995/TFG} para que aquellos que lo deseen puedan probar el proyecto.

\vspace{5cm}

\noindent \textbf{Fdo: \autor}

\vspace{2cm}

\begin{flushright}
\ciudad, a \today
\end{flushright}

\chapter*{}
\thispagestyle{empty}

\noindent\rule[-1ex]{\textwidth}{2pt}\\[4.5ex]

D. \textbf{\tutor}, profesor del \textbf{\departamento} de la \textbf{\universidad}.

\vspace{0.5cm}

\textbf{Informa:}

\vspace{0.5cm}

Que el presente trabajo, titulado \textit{\textbf{\titulo: \subtitulo}}, ha sido realizado bajo su supervisión por \textbf{\autor}, y autoriza la defensa de dicho trabajo ante el tribunal que corresponda.

\vspace{0.5cm}

Y para que conste, expide y firma el presente informe en \ciudad, a \today.

\vspace{1cm}

\textbf{El tutor:}

\vspace{5cm}

% \begin{figure}[H]
% \includegraphics[width=0.3\textwidth]{firma_tutor}
% \end{figure}

\noindent\textbf{\tutor}

\chapter*{Agradecimientos}
\thispagestyle{empty}

\vspace{1cm}

A mi tutor del TFG por toda la ayuda y apoyo durante el proyecto.\\

A mi familia por estar ahí siempre que lo he necesitado.\\

A mis amigos y amigas por todas las veces que me han ayudado y apoyado durante todas las etapas de mi vida.\\

A mis profesores por su incansable esfuerzo durante estos años de carrera para que aprendamos lo máximo posible y de la mejor manera.\\

\chapter{Proceso de desarrollo}
\label{ch:desarrollo}

\section{Metodologías de desarrollo ágil}

\section{Diseño centrado en el usuario}

\subsection{Historias de usuario}
En la Tabla \ref{tabla-lista-historias-usuario} podemos encontrar todas las historias de usuario.

%%%%%%%%%%%%%%%%%%%%%%%%%%%%%%%%%%%%%%%%%%%%%%%%%%TABLA LISTA HISTORIAS USUARIO%%%%%%%%%%%%%%%%%%%%%%%%%%%%%%%%%%%%%%%%%%%%%%%%%%%%%%

\begin{table}[h]
  \begin{center}
    \begin{tabular}{|p{1cm}|p{7.5cm}|p{1.9cm}|p{1.6cm}|}

      \hline
        \rowcolor{Gray} \textbf{ID}
        & \textbf{Título}
        & \textbf{Estimación}
        & \textbf{Prioridad}\\

      \hline
      HU1
      & Seleccionar personajes
      & 2
      & 1\\

      \hline
      HU2
      & Comenzar juego
      & 3
      & 1\\

      \hline
      HU3
      & Salir de la pantalla de instrucciones
      & 4
      & 1\\

      \hline
      HU4
      & Escanear tablero
      & 5
      & 1\\

      \hline
      HU5
      & Tirar los dados
      & 8
      & 1\\

      \hline
      HU6
      & Seleccionar habitación a la que desplazarse
      & 2
      & 1\\

      \hline
      HU7
      & Acceder al menú de anotaciones
      & 2
      & 1\\

      \hline
      HU8
      & Marcar personaje con una interrogación
      & 2
      & 1\\

      \hline
      HU9
      & Marcar arma con una interrogación
      & 1
      & 1\\

      \hline
      HU10
      & Marcar habitación con una interrogación
      & 1
      & 1\\

      \hline
      HU11
      & Marcar personaje con una X
      & 2
      & 1\\

      \hline
      HU12
      & Marcar arma con una X
      & 1
      & 1\\

      \hline
      HU13
      & Marcar habitación con una X
      & 1
      & 1\\

      \hline
      HU14
      & Escanear una acusación
      & 8
      & 1\\

      \hline
      HU15
      & Terminar partida
      & 2
      & 1\\

      \hline
      HU16
      & Cambiar de turno
      & 1
      & 1\\

      \hline

    \end{tabular}

    \caption{Listado de historias de usuario.}
    \label{tabla-lista-historias-usuario}

  \end{center}
\end{table}

En la Tabla \ref{tabla-product-backlog} podemos encontrar el Product Backlog.
%%%%%%%%%%%%%%%%%%%%%%%%%%%%%%%%%%%%%%%%%%%%%%%%%%TABLA PRODUCT BACKLOG%%%%%%%%%%%%%%%%%%%%%%%%%%%%%%%%%%%%%%%%%%%%%%%%%%%%%%

\begin{table}[h]
  \begin{center}
    \begin{tabular}{|p{1cm}|p{7.5cm}|p{1.9cm}|p{1.6cm}|p{1.6cm}|}

      \hline
        \rowcolor{Gray} \textbf{ID}
        & \textbf{Título}
        & \textbf{Estimación}
        & \textbf{Iteración}
        & \textbf{Entrega}\\

      \hline
      HU1
      & Seleccionar personajes
      & 2
      & 3
      & 2\\

      \hline
      HU2
      & Comenzar juego
      & 3
      & 3
      & 2\\

      \hline
      HU3
      & Salir de la pantalla de instrucciones
      & 4
      & 3
      & 2\\

      \hline
      HU4
      & Escanear tablero
      & 5
      & 4
      & 2\\

      \hline
      HU5
      & Tirar los dados
      & 8
      & 5
      & 3\\

      \hline
      HU6
      & Seleccionar habitación a la que desplazarse
      & 2
      & 5
      & 3\\

      \hline
      HU7
      & Acceder al menú de anotaciones
      & 2
      & 6
      & 3\\

      \hline
      HU8
      & Marcar personaje con una interrogación
      & 2
      & 6
      & 3\\

      \hline
      HU9
      & Marcar arma con una interrogación
      & 1
      & 6
      & 3\\

      \hline
      HU10
      & Marcar habitación con una interrogación
      & 1
      & 6
      & 3\\

      \hline
      HU11
      & Marcar personaje con una X
      & 2
      & 6
      & 3\\

      \hline
      HU12
      & Marcar arma con una X
      & 1
      & 6
      & 3\\

      \hline
      HU13
      & Marcar habitación con una X
      & 1
      & 6
      & 3\\

      \hline
      HU14
      & Escanear una acusación
      & 8
      & 7
      & 3\\

      \hline
      HU15
      & Terminar partida
      & 2
      & 4
      & 2\\

      \hline
      HU16
      & Cambiar de turno
      & 1
      & 8
      & 4\\

      \hline

    \end{tabular}

    \caption{Listado del Product Backlog.}
    \label{tabla-product-backlog}

  \end{center}
\end{table}

Desde la Tabla \ref{tabla-hu1} hasta la Tabla \ref{tabla-hu16} podemos encontrar las historias de usuario.

%%%%%%%%%%%%%%%%%%%%%%%%%%%%%%%%%%%%%%%%%%%%%%%%%%TABLAS HISOTIRAS DE USUARIO%%%%%%%%%%%%%%%%%%%%%%%%%%%%%%%%%%%%%%%%%%%%%%%%%%%%%%

\begin{table}[h]
  \begin{center}
    \begin{tabular}{|p{4cm}|p{4cm}|p{4cm}|}

    \hline
    \textbf{Identificador}: HU.1
    & \multicolumn{2}{p{8cm}|}{Seleccionar personajes}\\

    \hline
    \multicolumn{3}{|p{12cm}|}{\textbf{Descripción}: Como usuario jugador, quiero poder seleccionar un personaje de los disponibles en el juego.}\\

    \hline
    \textbf{Estimación}:2
    & \textbf{Prioridad}: 1
    & \textbf{Entrega}: 2\\

    \hline
    \multicolumn{3}{|p{12cm}|}{\textbf{Pruebas de aceptación}:
      \begin{itemize}
        \item Comprobar que los personajes elegidos se almacena correctamente.
      \end{itemize}
    }\\

    \hline

    \end{tabular}

    \caption{Tabla de la historia de usuario número 1.}
    \label{tabla-hu1}

  \end{center}
\end{table}

\begin{table}[h]
  \begin{center}
    \begin{tabular}{|p{4cm}|p{4cm}|p{4cm}|}

    \hline
    \textbf{Identificador}: HU.16
    & \multicolumn{2}{p{8cm}|}{Cambiar de turno}\\

    \hline
    \multicolumn{3}{|p{12cm}|}{\textbf{Descripción}: Como usuario jugador, quiero cambiar de turno al del otro jugador.}\\

    \hline
    \textbf{Estimación}:1
    & \textbf{Prioridad}: 1
    & \textbf{Entrega}: 4\\

    \hline
    \multicolumn{3}{|p{12cm}|}{\textbf{Pruebas de aceptación}:
      \begin{itemize}
        \item Comprobar que el menú de anotaciones que se muestra a los jugadores es diferente.
        \item Comprobar que la información que se muestra sobre el tablero es diferente para los diferentes jugadores.
        \item Comprobar que efectivamente estamos en el turno del otro jugador (comprobando que la información que se muestra es la del otro jugador).
      \end{itemize}
    }\\

    \hline

    \end{tabular}

    \caption{Tabla de la historia de usuario número 16.}
    \label{tabla-hu16}

  \end{center}
\end{table}

%%%%%%%%%%%%%%%%%%%%%%%%%%%%%%%%%%%%%%%%%%%%%%%%%% PLAN DE ENTREGAS %%%%%%%%%%%%%%%%%%%%%%%%%%%%%%%%%%%%%%%%%%%%%%%%%%%%%%

\section{Plan de entregas}

\textbf{Entrega 0}: Esta entrega consistirá en una evaluación de la viabilidad del proyecto en la que se desarrollará una aplicación que sea capaz de utilizando ARCore mostrar diferentes objetos 3D asociados a diferentes imágenes, cuando las escanea.

\begin{itemize}
  \item Fecha de entrega: 26/6/2018
\end{itemize}

\textbf{Entrega 1}: Esta entrega consistirá en bocetos a papel del juego, abarcando toda la funcionalidad de éste, y un informe que contendrá la información acerca de pruebas con diferentes usuarios sobre la usabilidad de la aplicación.

\begin{itemize}
  \item Fecha de entrega: 2/7/2018
\end{itemize}

\textbf{Entrega 2}: Esta entrega consistirá la aplicación que tendrá desarrollada la pantalla inicial de la aplicación, la pantalla de instrucciones y la pantalla de juego, en la que a partir del tablero se mostrarán los elementos (con realidad aumentada) necesarios para el juego.

\begin{itemize}
  \item Fecha de entrega: 16/7/2018
\end{itemize}

\textbf{Entrega 3}: Esta entrega consistirá en la aplicación a la que se habrá añadido la funcionalidad de tirar los dados, y por tanto mover al personaje a la habitación deseada, la funcionalidad de realizar una anotación, y la funcionalidad de hacer una acusación.

\begin{itemize}
  \item Fecha de entrega: 6/8/2018
\end{itemize}

\textbf{Entrega 4}: Esta entrega consistirá en la aplicación a la que se habrá añadido la funcionalidad de pasar de turno al del siguiente jugador, y por tanto mostrar la información correspondiente al otro jugador.

\begin{itemize}
  \item Fecha de entrega: 27/8/2018
\end{itemize}

%%%%%%%%%%%%%%%%%%%%%%%%%%%%%%%%%%%%%%%%%%%%%%%%%% BOCETOS %%%%%%%%%%%%%%%%%%%%%%%%%%%%%%%%%%%%%%%%%%%%%%%%%%%%%%
\section{Bocetos}
Se han llevado a cabo pruebas sobre los bocetos que se encuentran en la sección \ref{bocetos} del Anexo, dichas pruebas incluyen pruebas heurísticas realizadas por el desarrollador y pruebas se usabilidad con usuarios reales.

\subsection{Pruebas heurísticas}
\begin{itemize}

  \item \textbf{Principio 1: Visibilidad del estado del sistema.}\\
  La puntuación es de 7, el usuario está bien informado de lo que ocurre actualmente en el sistema, se muestra siempre que es necesario el botón “home” o el botón “atrás”, pero se puede mejorar, por ejemplo, indicando que se están escaneando imágenes mientras mueves el dispositivo sobre el tablero.

  \item \textbf{Principio 2: Correspondencia entre el sistema y el mundo real.}\\
  La puntuación es de 10, las opciones en los menús están ordenadas de forma lógica y el lenguaje que el juego utiliza es un lenguaje común al usuario del juego.

  \item \textbf{Principio 3: Control y libertad del usuario.}\\
  La puntuación es de 7, ya que en la mayoría de pantallas el usuario es libre de ir hacia adelante o hacia atrás, pero en instrucciones el usuario puede avanzar al juego pero no volver a la pantalla de inicio. Además, como se puede ver en la pantalla 6, para hacer apuntes, el botón de retroceso está en la zona derecha de la pantalla, pudiendo confundir esto al usuario, ya que es una función de retroceso no de avance.

  \item \textbf{Principio 4: Consistencia y estándares.}\\
  La puntuación es de 7, ya que es bastante consistente, pero en la pantalla que indica el ganador aparece el botón “home” como en otras pantallas, pero se muestra en una posición distinta, lo que resulta confuso al usuario habría que moverlo a la posición que ocupa siempre o utilizar otro botón.

  \item \textbf{Principio 5: Prevención de errores.}\\
  La puntuación es de 10, el diseño es bastante cuidadoso para la prevención de errores y el correcto tratamiento de estos.

  \item \textbf{Principio 6: Minimizar la carga de memoria del usuario.}\\
  La puntuación es de 10, el usuario no necesita recordar nada en ningún momento, todo se muestra de forma apropiada para que no suponga ninguna memorización al usuario.

  \item \textbf{Principio 7: Personalización y atajos.}
  La puntuación es de 10, la aplicación no dispone de personalización o atajos, pero no son necesarios en esta, por lo que no afecta a la experiencia de usuario.

  \item \textbf{Principio 8: Eficiencia de uso y rendimiento.}
  La puntuación es de 8, la aplicación está bien optimizada para que al usuario le resulte sencillo y rápido llevar a cabo cualquier tarea, pero si es cierto que algunos botones que se utilizan con mucha frecuencia no están en las posiciones óptimas.

  \item \textbf{Principio 9: Estética y diseño minimalista.}
  La puntuación es de 10, la información que se muestra en la aplicación es la necesaria para que el usuario pueda jugar con la mejor experiencia de usuario posible, no hay exceso o falta de información.

  \item \textbf{Principio 10: Ayuda al usuario a reconocer, diagnosticar y recuperarse de errores.}
  La puntuación es de 10, ya que no hay posibilidad de que ocurran errores en el juego.

  \item \textbf{Principio 11: Ayuda y documentación.}
  La puntuación es de 10, ya que antes de comenzar cada partida se muestra al usuario unas instrucciones de cómo funciona el juego.

  \item \textbf{Principio 12: Interacción física y ergonomía.}
  La puntuación es de 10, ya que los botones son fácilmente diferenciables y están en una posición cómoda para el usuario, teniendo en cuenta que al ser un juego utilizará las dos manos para usar el dispositivo móvil.

\end{itemize}

\subsection{Pruebas de usabilidad}
Las tablas que contienen la información obtenida en estas pruebas de usabilidad se encuentran en la sección \ref{tablas-usabilidad-bocetos} del Anexo.

\begin{itemize}
  \item \textbf{Usuario 1}

  \textbf{Pre Test}

  \begin{enumerate}
    \item Edad: 18
    \item Dispone de un dispositivo móvil: Sí
    \item Con qué frecuencia utiliza su dispositivo móvil: Varias veces al dia
    \item Con qué frecuencia juega a juegos de mesa: Varias veces al año
    \item Con qué frecuencia juega a juegos en su móvil: Varias veces a la semana
  \end{enumerate}

  \textbf{Test}: Los resultados del test de usabilidad sobre el Usuario 1 se encuentran en la Tabla \ref{tabla-bocetos-usuario1}


  \item \textbf{Usuario 2}

  \textbf{Pre Test}

  \begin{enumerate}
    \item Edad: 57
    \item Dispone de un dispositivo móvil: Sí
    \item Con qué frecuencia utiliza su dispositivo móvil: Varias veces al día
    \item Con qué frecuencia juega a juegos de mesa: Varias veces al año
    \item Con qué frecuencia juega a juegos en su móvil: Varias veces al día
  \end{enumerate}

  \textbf{Test}: Los resultados del test de usabilidad sobre el Usuario 2 se encuentran en la Tabla \ref{tabla-bocetos-usuario2}


  \item \textbf{Usuario 3}

  \textbf{Pre Test}

  \begin{enumerate}
    \item Edad: 26
    \item Dispone de un dispositivo móvil: Sí
    \item Con qué frecuencia utiliza su dispositivo móvil: Varias veces al dia
    \item Con qué frecuencia juega a juegos de mesa: Una vez al mes como máximo
    \item Con qué frecuencia juega a juegos en su móvil: Casi nunca
  \end{enumerate}

  \textbf{Test}: Los resultados del test de usabilidad sobre el Usuario 3 se encuentran en la Tabla \ref{tabla-bocetos-usuario3}

\subsection{Conclusiones}
Durante dichas pruebas hemos podido comprobar que la interfaz de la aplicación es amigable para los usuarios, que si bien han detectado algunos fallos, que serán solventados en el desarrollo del juego, por lo general han sabido realizar todas las tareas sin dificultad, desenvolviéndose con rapidez.\\

También se ha podido comprobar el interés de los usuarios por el funcionamiento de la realidad aumentada, resultándoles algo sorprendente y que sin duda tenían ganas de probar en las siguientes pruebas de usabilidad, lo que denota la esperada expectación de los usuarios de juegos y en general de dispositivos móviles sobre la realidad aumentada y las novedosas experiencias que esta aportará al ámbito de los dispositivos móviles.

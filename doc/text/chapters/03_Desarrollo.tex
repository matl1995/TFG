\chapter{Proceso de desarrollo}
\label{ch:desarrollo}

\section{Metodologías de desarrollo ágil}

\section{Diseño centrado en el usuario}

\subsection{Historias de usuario}
En la Tabla \ref{tabla-lista-historias-usuario} podemos encontrar todas las historias de usuario.

%%%%%%%%%%%%%%%%%%%%%%%%%%%%%%%%%%%%%%%%%%%%%%%%%%TABLA LISTA HISTORIAS USUARIO%%%%%%%%%%%%%%%%%%%%%%%%%%%%%%%%%%%%%%%%%%%%%%%%%%%%%%

\begin{table}[h]
  \begin{center}
    \begin{tabular}{|p{1cm}|p{7.5cm}|p{1.9cm}|p{1.6cm}|}

      \hline
        \rowcolor{Gray} \textbf{ID}
        & \textbf{Título}
        & \textbf{Estimación}
        & \textbf{Prioridad}\\

      \hline
      HU1
      & Seleccionar personajes
      & 2
      & 1\\

      \hline
      HU2
      & Comenzar juego
      & 3
      & 1\\

      \hline
      HU3
      & Salir de la pantalla de instrucciones
      & 4
      & 1\\

      \hline
      HU4
      & Escanear tablero
      & 5
      & 1\\

      \hline
      HU5
      & Tirar los dados
      & 8
      & 1\\

      \hline
      HU6
      & Seleccionar habitación a la que desplazarse
      & 2
      & 1\\

      \hline
      HU7
      & Acceder al menú de anotaciones
      & 2
      & 1\\

      \hline
      HU8
      & Marcar personaje con una interrogación
      & 2
      & 1\\

      \hline
      HU9
      & Marcar arma con una interrogación
      & 1
      & 1\\

      \hline
      HU10
      & Marcar habitación con una interrogación
      & 1
      & 1\\

      \hline
      HU11
      & Marcar personaje con una X
      & 2
      & 1\\

      \hline
      HU12
      & Marcar arma con una X
      & 1
      & 1\\

      \hline
      HU13
      & Marcar habitación con una X
      & 1
      & 1\\

      \hline
      HU14
      & Escanear una acusación
      & 8
      & 1\\

      \hline
      HU15
      & Terminar partida
      & 2
      & 1\\

      \hline
      HU16
      & Cambiar de turno
      & 1
      & 1\\

      \hline

    \end{tabular}

    \caption{Listado de historias de usuario.}
    \label{tabla-lista-historias-usuario}

  \end{center}
\end{table}

En la Tabla \ref{tabla-product-backlog} podemos encontrar el Product Backlog.
%%%%%%%%%%%%%%%%%%%%%%%%%%%%%%%%%%%%%%%%%%%%%%%%%%TABLA PRODUCT BACKLOG%%%%%%%%%%%%%%%%%%%%%%%%%%%%%%%%%%%%%%%%%%%%%%%%%%%%%%

\begin{table}[h]
  \begin{center}
    \begin{tabular}{|p{1cm}|p{7.5cm}|p{1.9cm}|p{1.6cm}|p{1.6cm}|}

      \hline
        \rowcolor{Gray} \textbf{ID}
        & \textbf{Título}
        & \textbf{Estimación}
        & \textbf{Iteración}
        & \textbf{Entrega}\\

      \hline
      HU1
      & Seleccionar personajes
      & 2
      & 3
      & 2\\

      \hline
      HU2
      & Comenzar juego
      & 3
      & 3
      & 2\\

      \hline
      HU3
      & Salir de la pantalla de instrucciones
      & 4
      & 3
      & 2\\

      \hline
      HU4
      & Escanear tablero
      & 5
      & 4
      & 2\\

      \hline
      HU5
      & Tirar los dados
      & 8
      & 5
      & 3\\

      \hline
      HU6
      & Seleccionar habitación a la que desplazarse
      & 2
      & 5
      & 3\\

      \hline
      HU7
      & Acceder al menú de anotaciones
      & 2
      & 6
      & 3\\

      \hline
      HU8
      & Marcar personaje con una interrogación
      & 2
      & 6
      & 3\\

      \hline
      HU9
      & Marcar arma con una interrogación
      & 1
      & 6
      & 3\\

      \hline
      HU10
      & Marcar habitación con una interrogación
      & 1
      & 6
      & 3\\

      \hline
      HU11
      & Marcar personaje con una X
      & 2
      & 6
      & 3\\

      \hline
      HU12
      & Marcar arma con una X
      & 1
      & 6
      & 3\\

      \hline
      HU13
      & Marcar habitación con una X
      & 1
      & 6
      & 3\\

      \hline
      HU14
      & Escanear una acusación
      & 8
      & 7
      & 3\\

      \hline
      HU15
      & Terminar partida
      & 2
      & 4
      & 2\\

      \hline
      HU16
      & Cambiar de turno
      & 1
      & 8
      & 4\\

      \hline

    \end{tabular}

    \caption{Listado del Product Backlog.}
    \label{tabla-product-backlog}

  \end{center}
\end{table}

Desde la Tabla \ref{tabla-hu1} hasta la Tabla \ref{tabla-hu16} podemos encontrar las historias de usuario.

%%%%%%%%%%%%%%%%%%%%%%%%%%%%%%%%%%%%%%%%%%%%%%%%%%TABLAS HISOTIRAS DE USUARIO%%%%%%%%%%%%%%%%%%%%%%%%%%%%%%%%%%%%%%%%%%%%%%%%%%%%%%

\begin{table}[h]
  \begin{center}
    \begin{tabular}{|p{4cm}|p{4cm}|p{4cm}|}

    \hline
    \textbf{Identificador}: HU.1
    & \multicolumn{2}{p{8cm}|}{Seleccionar personajes}\\

    \hline
    \multicolumn{3}{|p{12cm}|}{\textbf{Descripción}: Como usuario jugador, quiero poder seleccionar un personaje de los disponibles en el juego.}\\

    \hline
    \textbf{Estimación}:2
    & \textbf{Prioridad}: 1
    & \textbf{Entrega}: 2\\

    \hline
    \multicolumn{3}{|p{12cm}|}{\textbf{Pruebas de aceptación}:
      \begin{itemize}
        \item Comprobar que los personajes elegidos se almacena correctamente.
      \end{itemize}
    }\\

    \hline

    \end{tabular}

    \caption{Tabla de la historia de usuario número 1.}
    \label{tabla-hu1}

  \end{center}
\end{table}

\begin{table}[h]
  \begin{center}
    \begin{tabular}{|p{4cm}|p{4cm}|p{4cm}|}

    \hline
    \textbf{Identificador}: HU.2
    & \multicolumn{2}{p{8cm}|}{Comenzar juego}\\

    \hline
    \multicolumn{3}{|p{12cm}|}{\textbf{Descripción}: Como usuario jugador, quiero poder comenzar el juego una vez seleccionados los personajes.}\\

    \hline
    \textbf{Estimación}:3
    & \textbf{Prioridad}: 1
    & \textbf{Entrega}: 2\\

    \hline
    \multicolumn{3}{|p{12cm}|}{\textbf{Pruebas de aceptación}:
      \begin{itemize}
        \item Comprobar que el juego avanza a la siguiente pantalla posterior a la inicial.
      \end{itemize}
    }\\

    \hline

    \end{tabular}

    \caption{Tabla de la historia de usuario número 2.}
    \label{tabla-hu2}

  \end{center}
\end{table}

\begin{table}[h]
  \begin{center}
    \begin{tabular}{|p{4cm}|p{4cm}|p{4cm}|}

    \hline
    \textbf{Identificador}: HU.3
    & \multicolumn{2}{p{8cm}|}{Salir de la pantalla de instrucciones}\\

    \hline
    \multicolumn{3}{|p{12cm}|}{\textbf{Descripción}: Como usuario jugador, quiero poder avanzar a la siguiente pantalla después de haber leído las instrucciones.}\\

    \hline
    \textbf{Estimación}:4
    & \textbf{Prioridad}: 1
    & \textbf{Entrega}: 2\\

    \hline
    \multicolumn{3}{|p{12cm}|}{\textbf{Pruebas de aceptación}:
      \begin{itemize}
        \item Comprobar que el juego avanza a la siguiente pantalla posterior a la de instrucciones.
      \end{itemize}
    }\\

    \hline

    \end{tabular}

    \caption{Tabla de la historia de usuario número 3.}
    \label{tabla-hu3}

  \end{center}
\end{table}

\begin{table}[h]
  \begin{center}
    \begin{tabular}{|p{4cm}|p{4cm}|p{4cm}|}

    \hline
    \textbf{Identificador}: HU.4
    & \multicolumn{2}{p{8cm}|}{Escanear tablero}\\

    \hline
    \multicolumn{3}{|p{12cm}|}{\textbf{Descripción}: Como usuario jugador, quiero escanear el tablero para que comience el juego.}\\

    \hline
    \textbf{Estimación}:5
    & \textbf{Prioridad}: 1
    & \textbf{Entrega}: 2\\

    \hline
    \multicolumn{3}{|p{12cm}|}{\textbf{Pruebas de aceptación}:
      \begin{itemize}
        \item Comprobar que se reconoce la imagen de tablero.
        \item Comprobar que se muestra la información 3D relacionada al tablero de forma correcta.
        \item Comprobar que se muestran los botones necesarios para el juego una vez escaneado el tablero.
      \end{itemize}
    }\\

    \hline

    \end{tabular}

    \caption{Tabla de la historia de usuario número 4.}
    \label{tabla-hu4}

  \end{center}
\end{table}

\begin{table}[h]
  \begin{center}
    \begin{tabular}{|p{4cm}|p{4cm}|p{4cm}|}

    \hline
    \textbf{Identificador}: HU.5
    & \multicolumn{2}{p{8cm}|}{Tirar los dados}\\

    \hline
    \multicolumn{3}{|p{12cm}|}{\textbf{Descripción}: Como usuario jugador, quiero tirar los dados para obtener el número de movimientos que tengo.}\\

    \hline
    \textbf{Estimación}:8
    & \textbf{Prioridad}: 1
    & \textbf{Entrega}: 3\\

    \hline
    \multicolumn{3}{|p{12cm}|}{\textbf{Pruebas de aceptación}:
      \begin{itemize}
        \item Comprobar que el lanzamiento de dados es aleatorio.
        \item Comprobar que se realiza correctamente el efecto visual.
        \item Comprobar que el jugador que ha tirado los dados puede desplazarse hasta donde la tirada de dados le permita.
      \end{itemize}
    }\\

    \hline

    \end{tabular}

    \caption{Tabla de la historia de usuario número 5.}
    \label{tabla-hu5}

  \end{center}
\end{table}

\begin{table}[h]
  \begin{center}
    \begin{tabular}{|p{4cm}|p{4cm}|p{4cm}|}

    \hline
    \textbf{Identificador}: HU.6
    & \multicolumn{2}{p{8cm}|}{Seleccionar habitación a la que desplazarse}\\

    \hline
    \multicolumn{3}{|p{12cm}|}{\textbf{Descripción}: Como usuario jugador, quiero poder seleccionar la habitación a la que desplazarme en función del número obtenido en el lanzamiento de dados.}\\

    \hline
    \textbf{Estimación}:2
    & \textbf{Prioridad}: 1
    & \textbf{Entrega}: 3\\

    \hline
    \multicolumn{3}{|p{12cm}|}{\textbf{Pruebas de aceptación}:
      \begin{itemize}
        \item Comprobar que el personaje se desplaza a la habitación seleccionada.
      \end{itemize}
    }\\

    \hline

    \end{tabular}

    \caption{Tabla de la historia de usuario número 6.}
    \label{tabla-hu6}

  \end{center}
\end{table}

\begin{table}[h]
  \begin{center}
    \begin{tabular}{|p{4cm}|p{4cm}|p{4cm}|}

    \hline
    \textbf{Identificador}: HU.7
    & \multicolumn{2}{p{8cm}|}{Acceder al menú de anotaciones}\\

    \hline
    \multicolumn{3}{|p{12cm}|}{\textbf{Descripción}: Como usuario jugador, quiero acceder al menú de notas, para poder apuntar mi información.}\\

    \hline
    \textbf{Estimación}:2
    & \textbf{Prioridad}: 1
    & \textbf{Entrega}: 3\\

    \hline
    \multicolumn{3}{|p{12cm}|}{\textbf{Pruebas de aceptación}:
      \begin{itemize}
        \item Comprobar que se muestra el menú correctamente.
      \end{itemize}
    }\\

    \hline

    \end{tabular}

    \caption{Tabla de la historia de usuario número 7.}
    \label{tabla-hu7}

  \end{center}
\end{table}

\begin{table}[h]
  \begin{center}
    \begin{tabular}{|p{4cm}|p{4cm}|p{4cm}|}

    \hline
    \textbf{Identificador}: HU.8
    & \multicolumn{2}{p{8cm}|}{Marcar personaje con una interrogación}\\

    \hline
    \multicolumn{3}{|p{12cm}|}{\textbf{Descripción}: Como usuario jugador, quiero marcar un personaje con una interrogación.}\\

    \hline
    \textbf{Estimación}:2
    & \textbf{Prioridad}: 1
    & \textbf{Entrega}: 3\\

    \hline
    \multicolumn{3}{|p{12cm}|}{\textbf{Pruebas de aceptación}:
      \begin{itemize}
        \item Comprobar que se marca el personaje indicado con una interrogación.
      \end{itemize}
    }\\

    \hline

    \end{tabular}

    \caption{Tabla de la historia de usuario número 8.}
    \label{tabla-hu8}

  \end{center}
\end{table}

\begin{table}[h]
  \begin{center}
    \begin{tabular}{|p{4cm}|p{4cm}|p{4cm}|}

    \hline
    \textbf{Identificador}: HU.9
    & \multicolumn{2}{p{8cm}|}{Marcar arma con una interrogación}\\

    \hline
    \multicolumn{3}{|p{12cm}|}{\textbf{Descripción}: Como usuario jugador, quiero marcar un arma con una interrogación.}\\

    \hline
    \textbf{Estimación}:1
    & \textbf{Prioridad}: 1
    & \textbf{Entrega}: 3\\

    \hline
    \multicolumn{3}{|p{12cm}|}{\textbf{Pruebas de aceptación}:
      \begin{itemize}
        \item Comprobar que se marca el arma indicada con una interrogación.
      \end{itemize}
    }\\

    \hline

    \end{tabular}

    \caption{Tabla de la historia de usuario número 9.}
    \label{tabla-hu9}

  \end{center}
\end{table}

\begin{table}[h]
  \begin{center}
    \begin{tabular}{|p{4cm}|p{4cm}|p{4cm}|}

    \hline
    \textbf{Identificador}: HU.10
    & \multicolumn{2}{p{8cm}|}{Marcar habitación con una interrogación}\\

    \hline
    \multicolumn{3}{|p{12cm}|}{\textbf{Descripción}: Como usuario jugador, quiero marcar una habitación con una interrogación.}\\

    \hline
    \textbf{Estimación}:1
    & \textbf{Prioridad}: 1
    & \textbf{Entrega}: 3\\

    \hline
    \multicolumn{3}{|p{12cm}|}{\textbf{Pruebas de aceptación}:
      \begin{itemize}
        \item Comprobar que se marca la habitación indicada con una interrogación.
      \end{itemize}
    }\\

    \hline

    \end{tabular}

    \caption{Tabla de la historia de usuario número 10.}
    \label{tabla-hu10}

  \end{center}
\end{table}

\begin{table}[h]
  \begin{center}
    \begin{tabular}{|p{4cm}|p{4cm}|p{4cm}|}

    \hline
    \textbf{Identificador}: HU.11
    & \multicolumn{2}{p{8cm}|}{Marcar personaje con una X}\\

    \hline
    \multicolumn{3}{|p{12cm}|}{\textbf{Descripción}: Como usuario jugador, quiero marcar un personaje con una X.}\\

    \hline
    \textbf{Estimación}:2
    & \textbf{Prioridad}: 1
    & \textbf{Entrega}: 3\\

    \hline
    \multicolumn{3}{|p{12cm}|}{\textbf{Pruebas de aceptación}:
      \begin{itemize}
        \item Comprobar que se marca el personaje indicado con una X.
      \end{itemize}
    }\\

    \hline

    \end{tabular}

    \caption{Tabla de la historia de usuario número 11.}
    \label{tabla-hu11}

  \end{center}
\end{table}

\begin{table}[h]
  \begin{center}
    \begin{tabular}{|p{4cm}|p{4cm}|p{4cm}|}

    \hline
    \textbf{Identificador}: HU.12
    & \multicolumn{2}{p{8cm}|}{Marcar arma con una X}\\

    \hline
    \multicolumn{3}{|p{12cm}|}{\textbf{Descripción}: Como usuario jugador, quiero marcar un arma con una X.}\\

    \hline
    \textbf{Estimación}:1
    & \textbf{Prioridad}: 1
    & \textbf{Entrega}: 3\\

    \hline
    \multicolumn{3}{|p{12cm}|}{\textbf{Pruebas de aceptación}:
      \begin{itemize}
        \item Comprobar que se marca el arma indicada con una X.
      \end{itemize}
    }\\

    \hline

    \end{tabular}

    \caption{Tabla de la historia de usuario número 12.}
    \label{tabla-hu12}

  \end{center}
\end{table}

\begin{table}[h]
  \begin{center}
    \begin{tabular}{|p{4cm}|p{4cm}|p{4cm}|}

    \hline
    \textbf{Identificador}: HU.13
    & \multicolumn{2}{p{8cm}|}{Marcar habitación con una X}\\

    \hline
    \multicolumn{3}{|p{12cm}|}{\textbf{Descripción}: Como usuario jugador, quiero marcar una habitación con una X.}\\

    \hline
    \textbf{Estimación}:1
    & \textbf{Prioridad}: 1
    & \textbf{Entrega}: 3\\

    \hline
    \multicolumn{3}{|p{12cm}|}{\textbf{Pruebas de aceptación}:
      \begin{itemize}
        \item Comprobar que se marca la habitación indicada con una X.
      \end{itemize}
    }\\

    \hline

    \end{tabular}

    \caption{Tabla de la historia de usuario número 13.}
    \label{tabla-hu13}

  \end{center}
\end{table}

\begin{table}[h]
  \begin{center}
    \begin{tabular}{|p{4cm}|p{4cm}|p{4cm}|}

    \hline
    \textbf{Identificador}: HU.14
    & \multicolumn{2}{p{8cm}|}{Escanear una acusación}\\

    \hline
    \multicolumn{3}{|p{12cm}|}{\textbf{Descripción}: Como usuario jugador, quiero escanear las cartas para realizar una acusación.}\\

    \hline
    \textbf{Estimación}:8
    & \textbf{Prioridad}: 1
    & \textbf{Entrega}: 3\\

    \hline
    \multicolumn{3}{|p{12cm}|}{\textbf{Pruebas de aceptación}:
      \begin{itemize}
        \item Comprobar que se reconocen las cartas que forman parte de la acusación.
        \item Comprobar que la comprobación de si es la solución se hace de forma correcta.
        \item Comprobar que si la acusación es incorrecta pase al turno del siguiente jugador.
        \item Comprobar que si la acusación es correcta pase a la pantalla de victoria.
      \end{itemize}
    }\\

    \hline

    \end{tabular}

    \caption{Tabla de la historia de usuario número 14.}
    \label{tabla-hu14}

  \end{center}
\end{table}

\begin{table}[h]
  \begin{center}
    \begin{tabular}{|p{4cm}|p{4cm}|p{4cm}|}

    \hline
    \textbf{Identificador}: HU.15
    & \multicolumn{2}{p{8cm}|}{Terminar partida}\\

    \hline
    \multicolumn{3}{|p{12cm}|}{\textbf{Descripción}: Como usuario jugador, quiero terminar la partida.}\\

    \hline
    \textbf{Estimación}: 2
    & \textbf{Prioridad}: 1
    & \textbf{Entrega}: 3\\

    \hline
    \multicolumn{3}{|p{12cm}|}{\textbf{Pruebas de aceptación}:
      \begin{itemize}
        \item Comprobar que el juego cambia a la pantalla de inicio.
      \end{itemize}
    }\\

    \hline

    \end{tabular}

    \caption{Tabla de la historia de usuario número 15.}
    \label{tabla-hu15}

  \end{center}
\end{table}

\begin{table}[h]
  \begin{center}
    \begin{tabular}{|p{4cm}|p{4cm}|p{4cm}|}

    \hline
    \textbf{Identificador}: HU.16
    & \multicolumn{2}{p{8cm}|}{Cambiar de turno}\\

    \hline
    \multicolumn{3}{|p{12cm}|}{\textbf{Descripción}: Como usuario jugador, quiero cambiar de turno al del otro jugador.}\\

    \hline
    \textbf{Estimación}:1
    & \textbf{Prioridad}: 1
    & \textbf{Entrega}: 4\\

    \hline
    \multicolumn{3}{|p{12cm}|}{\textbf{Pruebas de aceptación}:
      \begin{itemize}
        \item Comprobar que el menú de anotaciones que se muestra a los jugadores es diferente.
        \item Comprobar que la información que se muestra sobre el tablero es diferente para los diferentes jugadores.
        \item Comprobar que efectivamente estamos en el turno del otro jugador (comprobando que la información que se muestra es la del otro jugador).
      \end{itemize}
    }\\

    \hline

    \end{tabular}

    \caption{Tabla de la historia de usuario número 16.}
    \label{tabla-hu16}

  \end{center}
\end{table}

\FloatBarrier

%%%%%%%%%%%%%%%%%%%%%%%%%%%%%%%%%%%%%%%%%%%%%%%%%% PLAN DE ENTREGAS %%%%%%%%%%%%%%%%%%%%%%%%%%%%%%%%%%%%%%%%%%%%%%%%%%%%%%

\section{Plan de entregas}

\textbf{Entrega 0}: Esta entrega consistirá en una evaluación de la viabilidad del proyecto en la que se desarrollará una aplicación que sea capaz de utilizando ARCore mostrar diferentes objetos 3D asociados a diferentes imágenes, cuando las escanea.

\begin{itemize}
  \item Fecha de entrega: 26/6/2018
\end{itemize}

\textbf{Entrega 1}: Esta entrega consistirá en bocetos a papel del juego, abarcando toda la funcionalidad de éste, y un informe que contendrá la información acerca de pruebas con diferentes usuarios sobre la usabilidad de la aplicación.

\begin{itemize}
  \item Fecha de entrega: 2/7/2018
\end{itemize}

\textbf{Entrega 2}: Esta entrega consistirá la aplicación que tendrá desarrollada la pantalla inicial de la aplicación, la pantalla de instrucciones y la pantalla de juego, en la que a partir del tablero se mostrarán los elementos (con realidad aumentada) necesarios para el juego.

\begin{itemize}
  \item Fecha de entrega: 16/7/2018
\end{itemize}

\textbf{Entrega 3}: Esta entrega consistirá en la aplicación a la que se habrá añadido la funcionalidad de tirar los dados, y por tanto mover al personaje a la habitación deseada, la funcionalidad de realizar una anotación, y la funcionalidad de hacer una acusación.

\begin{itemize}
  \item Fecha de entrega: 6/8/2018
\end{itemize}

\textbf{Entrega 4}: Esta entrega consistirá en la aplicación a la que se habrá añadido la funcionalidad de pasar de turno al del siguiente jugador, y por tanto mostrar la información correspondiente al otro jugador.

\begin{itemize}
  \item Fecha de entrega: 27/8/2018
\end{itemize}

%%%%%%%%%%%%%%%%%%%%%%%%%%%%%%%%%%%%%%%%%%%%%%%%%% BOCETOS %%%%%%%%%%%%%%%%%%%%%%%%%%%%%%%%%%%%%%%%%%%%%%%%%%%%%%
\section{Bocetos}
Se han llevado a cabo pruebas sobre los bocetos que se encuentran en la sección \ref{bocetos} del Anexo, dichas pruebas incluyen pruebas heurísticas realizadas por el desarrollador y pruebas se usabilidad con usuarios reales.

\subsection{Pruebas heurísticas}
\begin{itemize}

  \item \textbf{Principio 1: Visibilidad del estado del sistema.}\\
  La puntuación es de 7, el usuario está bien informado de lo que ocurre actualmente en el sistema, se muestra siempre que es necesario el botón “home” o el botón “atrás”, pero se puede mejorar, por ejemplo, indicando que se están escaneando imágenes mientras mueves el dispositivo sobre el tablero.

  \item \textbf{Principio 2: Correspondencia entre el sistema y el mundo real.}\\
  La puntuación es de 10, las opciones en los menús están ordenadas de forma lógica y el lenguaje que el juego utiliza es un lenguaje común al usuario del juego.

  \item \textbf{Principio 3: Control y libertad del usuario.}\\
  La puntuación es de 7, ya que en la mayoría de pantallas el usuario es libre de ir hacia adelante o hacia atrás, pero en instrucciones el usuario puede avanzar al juego pero no volver a la pantalla de inicio. Además, como se puede ver en la pantalla 6, para hacer apuntes, el botón de retroceso está en la zona derecha de la pantalla, pudiendo confundir esto al usuario, ya que es una función de retroceso no de avance.

  \item \textbf{Principio 4: Consistencia y estándares.}\\
  La puntuación es de 7, ya que es bastante consistente, pero en la pantalla que indica el ganador aparece el botón “home” como en otras pantallas, pero se muestra en una posición distinta, lo que resulta confuso al usuario habría que moverlo a la posición que ocupa siempre o utilizar otro botón.

  \item \textbf{Principio 5: Prevención de errores.}\\
  La puntuación es de 10, el diseño es bastante cuidadoso para la prevención de errores y el correcto tratamiento de estos.

  \item \textbf{Principio 6: Minimizar la carga de memoria del usuario.}\\
  La puntuación es de 10, el usuario no necesita recordar nada en ningún momento, todo se muestra de forma apropiada para que no suponga ninguna memorización al usuario.

  \item \textbf{Principio 7: Personalización y atajos.}
  La puntuación es de 10, la aplicación no dispone de personalización o atajos, pero no son necesarios en esta, por lo que no afecta a la experiencia de usuario.

  \item \textbf{Principio 8: Eficiencia de uso y rendimiento.}
  La puntuación es de 8, la aplicación está bien optimizada para que al usuario le resulte sencillo y rápido llevar a cabo cualquier tarea, pero si es cierto que algunos botones que se utilizan con mucha frecuencia no están en las posiciones óptimas.

  \item \textbf{Principio 9: Estética y diseño minimalista.}
  La puntuación es de 10, la información que se muestra en la aplicación es la necesaria para que el usuario pueda jugar con la mejor experiencia de usuario posible, no hay exceso o falta de información.

  \item \textbf{Principio 10: Ayuda al usuario a reconocer, diagnosticar y recuperarse de errores.}
  La puntuación es de 10, ya que no hay posibilidad de que ocurran errores en el juego.

  \item \textbf{Principio 11: Ayuda y documentación.}
  La puntuación es de 10, ya que antes de comenzar cada partida se muestra al usuario unas instrucciones de cómo funciona el juego.

  \item \textbf{Principio 12: Interacción física y ergonomía.}
  La puntuación es de 10, ya que los botones son fácilmente diferenciables y están en una posición cómoda para el usuario, teniendo en cuenta que al ser un juego utilizará las dos manos para usar el dispositivo móvil.

\end{itemize}

\subsection{Pruebas de usabilidad}
Las tablas que contienen la información obtenida en estas pruebas de usabilidad se encuentran en la sección \ref{tablas-usabilidad-bocetos} del Anexo.

\begin{itemize}
  \item \textbf{Usuario 1}

  \textbf{Pre Test}

  \begin{enumerate}
    \item Edad: 18
    \item Dispone de un dispositivo móvil: Sí
    \item Con qué frecuencia utiliza su dispositivo móvil: Varias veces al dia
    \item Con qué frecuencia juega a juegos de mesa: Varias veces al año
    \item Con qué frecuencia juega a juegos en su móvil: Varias veces a la semana
  \end{enumerate}

  \textbf{Test}: Los resultados del test de usabilidad sobre el Usuario 1 se encuentran en la Tabla \ref{tabla-bocetos-usuario1}


  \item \textbf{Usuario 2}

  \textbf{Pre Test}

  \begin{enumerate}
    \item Edad: 57
    \item Dispone de un dispositivo móvil: Sí
    \item Con qué frecuencia utiliza su dispositivo móvil: Varias veces al día
    \item Con qué frecuencia juega a juegos de mesa: Varias veces al año
    \item Con qué frecuencia juega a juegos en su móvil: Varias veces al día
  \end{enumerate}

  \textbf{Test}: Los resultados del test de usabilidad sobre el Usuario 2 se encuentran en la Tabla \ref{tabla-bocetos-usuario2}


  \item \textbf{Usuario 3}

  \textbf{Pre Test}

  \begin{enumerate}
    \item Edad: 26
    \item Dispone de un dispositivo móvil: Sí
    \item Con qué frecuencia utiliza su dispositivo móvil: Varias veces al dia
    \item Con qué frecuencia juega a juegos de mesa: Una vez al mes como máximo
    \item Con qué frecuencia juega a juegos en su móvil: Casi nunca
  \end{enumerate}

  \textbf{Test}: Los resultados del test de usabilidad sobre el Usuario 3 se encuentran en la Tabla \ref{tabla-bocetos-usuario3}

\subsection{Conclusiones}
Durante dichas pruebas hemos podido comprobar que la interfaz de la aplicación es amigable para los usuarios, que si bien han detectado algunos fallos, que serán solventados en el desarrollo del juego, por lo general han sabido realizar todas las tareas sin dificultad, desenvolviéndose con rapidez.\\

También se ha podido comprobar el interés de los usuarios por el funcionamiento de la realidad aumentada, resultándoles algo sorprendente y que sin duda tenían ganas de probar en las siguientes pruebas de usabilidad, lo que denota la esperada expectación de los usuarios de juegos y en general de dispositivos móviles sobre la realidad aumentada y las novedosas experiencias que esta aportará al ámbito de los dispositivos móviles.

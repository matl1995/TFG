\chapter{Análisis inicial del problema}
\label{ch:analisis}

\section{Concepto inicial del juego}
El juego contará con 2 jugadores.\\

Para empezar hay 3 cartas que contienen quien lo hizo, con que arma y donde, estas serán establecidas de forma aleatoria.\\

Hay una distancia x entre habitaciones, por la que podrás ir de una habitación a otra si obtienes esa distancia, en caso de obtener una distancia que no permite ir a ninguna habitación, esta se almacena y en el siguiente turno cuando tires al dado se indicará a qué habitaciones puedes ir.\\

Para hacer acusación tienes que poner en fila las 3 cartas de la solución y escanearlas, en caso de que sean correctas has ganado, en otro caso se indica que has fallado y se cambia al turno del siguiente jugador.\\

Para marcar tu información para ir averiguando quien es el asesino se pueden tomar notas, se muestra al usuario una lista de personajes, ubicaciones y armas, que va vas seleccionando para tachar o poner en interrogación, y dejando libre los que pueden ser la solución. Estas listas se muestran al pulsar un botón que se muestre en pantalla con el texto "Notas".\\

En pantalla se mostrarán, sin ocupar mucho espacio en pantalla cuatro botones:
\begin{itemize}
  \item \textbf{Primer botón}: Para tirar el dado, esto nos devolverá un número con las casillas que podemos recorrer.
  \item \textbf{Segundo botón}: Pasar turno, al pulsarlo es el turno del siguiente jugador.
  \item \textbf{Tercer botón}: Apuntar, aparece una lista donde se puede marcar que personajes, armas o habitaciones no forman parte del asesinato, o cuales se tienen en duda.
  \item \textbf{Cuarto botón}: Un botón “home” que permite salir de la partida actual.
\end{itemize}

Las cartas que te son asignadas se mostrarán debajo del tablero, en cada habitación se mostrará un modelo 3D del personaje y de las armas que hay en dicha habitación.\\

Los peones de los jugadores se establecen en habitaciones aleatorias.\\

Cada turno o dos turnos se muestra una pista al jugador que será sobre la habitación actual en la que se encuentra, los personajes que contiene esa habitación o las armas que contiene esa habitación.\\

\section{Narrativa}
El jugador es un detective que tiene que resolver un asesinato, mediante el desplazamiento a habitaciones irá recibiendo pistas de que elementos forman o no parte de la solución, y mediante acusaciones irá descartando o acertando elementos que forman parte de dicha solución.

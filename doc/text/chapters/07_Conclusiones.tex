\chapter{Conclusiones y Trabajos futuros}
\label{ch:conclusiones}
Este capítulo recoge las conclusiones con respecto al proyecto desarrollado y el proceso de desarrollo. También se evaluacia el cumplimiento de los objetivos establecidos en el Capítulo \ref{ch:introduccion}. Y por último, se expondrán las propuestas de mejora del proyecto.

\section{Conclusiones}
En este proyecto se ha trabajado en el desarrollo de un juego de tablero que se basa en el uso de tecnologías de realidad aumentada
y para ello se ha elegido ARCore, ya que es un SDK bastante nuevo pero bastante prometedor.

Una vez conlcuido el proyecto, el resultado que se ha obtenido ha sido satisfactorio, al haberse desarrollado un juego de tablero completamente funcional en el tiempo deseado.

Con respecto a la aplicación de las tecnologías de realidad aumentada en juegos de tablero, se ha podido comprobar que estas pueden ofrecer un buen resultado, dejando de lado la forma tradicional de digitalizar los juegos de mesa, y generando un balance entre elementos físicos y virtuales que ofrece una experiencia realista a la vez que espectacular. Esto puede suponer un antes y un despues para la digitalización de juegos de mesa, haciendolos realmente atractivos para los amantes de estos y para las nuevas generaciones.

Con respecto a ARCore se ha podido observar que todavía tiene un largo camino por delante, si bien es bastante funcional tiene ciertos aspectos que tienen mucho que mejorar. La detección de imagenes funciona pero con problemas, necesitando imagenes muy grandes y tardando en detectarlas, si bien funciona con imagenes del tamaño de una carta, hay dificultades cada vez que una de estas se intenta reconocer.

Por otro lado también se ha podido comprobar que la comunidad de ARCore aun es pequeña debido a ser un SDK tan nuevo, lo que genera dificultades a la hora de resolver errores, ya que la mayoría de desarrolladores en ARCore se están iniciando actualmente.

La experiencia de planificación y desarrollo de un proyecto utilizando metodologías ágiles y diseño centrado en el usuario ha sido sin duda enriquecedora, permitiendome adentrarme de forma realista en el desarrollo de un proyecto, y brindandome conocimientos y aptitudes a la hora del desarrollo de estos.

Tras la realización del proyecto se concluye que los objetivos establecidos en el Capítulo \ref{ch:introduccion} se han cumplido de forma existosa. A continuación se va a evaluar el cumplimiento de los objetivos de forma mas concreta:

\begin{itemize}
  \item La realización del proyecto me ha permitido ver que puede aportar la realidad aumentada a los juegos de tablero en dispositivos móviles, que como ya he expuesto en las conlcusiones puede ser el futuro de los juegos de mesa.El objetivo principal de este proyecto es explorar que ventajas puede aportar la realidad aumentada, mas concretamente el SDK ARCore, a los juegos en dispositivos móviles, y mas específicamente a los juegos de mesa en dispositivos móviles. Por tanto, mediante este proyecto se adquirirá experiencia en el desarrollo con tecnologías de realidad aumentada.

  \item He adquirido experiencia en el desarrollo de aplicaciones móviles con realidad aumentada, mas concretamente con ARCore, desenvolviendome con soltura con la librería que este ofrece.

  \item Se ha investigado en una alternativa a la tradicional realización de juegos de mesa en dispositivos móviles, transformando el enfoque habitual de tenerlo todo en la pantalla que provocaba una perdida de atractivo en los juegos de mesa, a un balance entre elementos virtuales y físicos que mantiene el realismo de los juegos de mesa.

  \item Se han adquirido conocimientos en la planificación y desarrollo de proyectos utilizando metodologías ágiles, al haber realizado todo el proyecto con dichos métodos.

  \item He adquirido un alto nivel de conocimiento sobre Unity, partía de no haber utilizado Unity antes, y durante el proyecto he aprendido el funcionamiento de la herramienta, y de como esta funcionaba en conjunto con ARCore, de forma que ahora tengo los conocimientos para manejar la herramienta y desarrollar otros proyectos con facilidad.

  \item He adquirido los conocimientos sobre el funcionamiento de la realidad aumentada y los elementos que esta utiliza para asociar información virtual a una escena o elemento.

  \item He explorado las diferentes alternativas a ARCore y lo que estas ofrecen, lo que me ha permitido ver las ventajas e inconvenientes de estas, y comprender las posibilidades que existen a la hora de realizar un proyecto con realidad aumentada.

  \item He adquirido los conocimientos para utilizar ARCore en la herramienta Unity para el desarrollo de aplicaciones de realidad aumentada.

\end{itemize}

\section{Trabajo futuro}
Se han evaluado diferentes opciones para el trabajo a realizar una vez finalice el proyecto, y se han concluido las siguientes opciones:

\begin{itemize}
  \item La ampliación del número de jugadores de 2 a 6, de forma que se adapte en mayor grado a un juego de mesa tradicional que suele incluir un mayor número de jugadores, fomentando así el factor social de los juegos de mesa.

  \item La extensión del juego a diferentes dispositivos, de forma que se permita a cada jugador utilizar su dispositivo móvil en lugar de compartir un dispositivo, esto requeriría la utilización de ``Cloud anchors", una funcionalidad que ARCore ofrece para compartir las coordenadas de la información asociada al mundo real entre dispositivos.

  \item La adición de animaciones y secuencias cinemáticas que añadan mas espectacularidad visual al juego, permitiendo así una mejorada experiencia de usuario.

  \item La sustitución de los modelos 3D utilizados en el juego que han sido obtenidos en la Unity Assets Store de forma gratuita por modelos 3D específicamente desarrollados para el juego, que encajen mejor en el ambiente de misterio y la estética que se describe en el GDD.

  \item La exploración de ``Cloud Anchors", una funcionalidad que ARCore ofrece para compartir las coordenadas de la información asociada al mundo real entre dispositivos, de forma que cuando los otros usuarios analicen la misma escena verán la misma información asociada a la escena. Esto se utilizaría para la extensión del juego a diferentes dispositivos, de forma que visualizando la misma escena cada uno puede jugar en su dispositivo, evitando así la incomodidad de tener que compartir el mismo dispositivo entre jugadores.

\end{itemize}

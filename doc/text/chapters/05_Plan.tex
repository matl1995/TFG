\chapter{Plan de entregas}
\label{ch:plan}
En este capítulo se exponen el plan de entregas, que será el elemento central de la planificación del proyecto, cada una de estas entregas contendrá los resultados a entregar e iteraciones, estas iteraciones a su vez, se componen por tareas que servirán para obetener esos resultados para la fecha de entrega. También se encuentran en este capítulo las historias de usuario, que mediante situaciones en las que se puede encontrar el usuario en el juego, servirán para establecer los requisitos este.

%%%%%%%%%%%%%%%%%%%%%%%%%%%%%%%%%%%%%%%%%%%%%%%%%% PLAN DE ENTREGAS %%%%%%%%%%%%%%%%%%%%%%%%%%%%%%%%%%%%%%%%%%%%%%%%%%%%%%

\section{Plan de entregas}
Este plan está formado por entregas, y a su vez cada entrega se compone de iteraciones. Las entregas tendrán una descripción, que consiste en las tareas que formarán parte de dicha entrega, y por tanto deberán estar completados para la fecha de entrega. Cada iteración tiene a su vez tareas que hay que llevar a cabo dentro de un plazo.\\

\textbf{Entrega 0}: Esta entrega consistirá en una evaluación de la viabilidad del proyecto en la que se desarrollará una aplicación que sea capaz de utilizando ARCore mostrar diferentes objetos 3D asociados a diferentes imágenes, cuando las escanea.

\begin{itemize}
  \item Fecha de entrega: 26/6/2018
\end{itemize}

\hfill

Este entrega se compone de las siguientes iteraciones:

\begin{itemize}
  \item \textbf{Primera Iteración}: En esta iteración se trabajará la realización de una aplicación con ARCore que utilice dicha tecnología para mostrar elementos 3D asociados a diferentes imágenes al mismo tiempo, que servirá como prueba de viabilidad del proyecto, se ha construido la versión inicial del GDD, se han creado las historias de usuario y, por último, se ha llevado a cabo la búsqueda y obtención de los diferentes modelos 3D que se van a utilizar en la aplicación.

  \begin{itemize}
    \item Fecha de comienzo: 15/6/2018
    \item Fecha de finalización: 26/6/2018
  \end{itemize}

\end{itemize}

\hfill

\textbf{Entrega 1}: Esta entrega consistirá en bocetos a papel del juego, abarcando toda la funcionalidad de éste, y un informe que contendrá la información acerca de pruebas con diferentes usuarios sobre la usabilidad de la aplicación.

\begin{itemize}
  \item Fecha de entrega: 2/7/2018
\end{itemize}

\hfill

Este entrega se compone de las siguientes iteraciones:

\begin{itemize}
  \item \textbf{Segunda Iteración}: En esta iteración se trabajarán los bocetos en papel del juego y las pruebas con diferentes usuarios sobre la usabilidad de la aplicación.

  \begin{itemize}
    \item Fecha de comienzo: 26/6/2018
    \item Fecha de finalización: 2/7/2018
  \end{itemize}
\end{itemize}

\hfill

\textbf{Entrega 2}: Esta entrega consistirá la aplicación que tendrá desarrollada la pantalla inicial de la aplicación, la pantalla de instrucciones y la pantalla de juego, en la que a partir del tablero se mostrarán los elementos (con realidad aumentada) necesarios para el juego.

\begin{itemize}
  \item Fecha de entrega: 16/7/2018
\end{itemize}

\hfill

Este entrega se compone de las siguientes iteraciones:

\begin{itemize}
  \item \textbf{Tercera Iteración}: En esta iteración se trabajará en la pantalla inicial de la aplicación y la pantalla de instrucciones.

  \begin{itemize}
    \item Fecha de comienzo: 2/7/2018
    \item Fecha de finalización: 9/7/2018
  \end{itemize}

  \item \textbf{Cuarta Iteración}: En esta iteración se trabajará en la pantalla de juego, en la que a partir del tablero se mostrarán los elementos (con realidad aumentada) necesarios para el juego.

  \begin{itemize}
    \item Fecha de comienzo: 9/7/2018
    \item Fecha de finalización: 16/7/2018
  \end{itemize}
\end{itemize}

\hfill

\textbf{Entrega 3}: Esta entrega consistirá en la aplicación a la que se habrá añadido la funcionalidad de tirar los dados, y por tanto mover al personaje a la habitación deseada, la funcionalidad de realizar una anotación, y la funcionalidad de hacer una acusación.

\begin{itemize}
  \item Fecha de entrega: 6/8/2018
\end{itemize}

\hfill

Este entrega se compone de las siguientes iteraciones:

\begin{itemize}
  \item \textbf{Quinta Iteración}: En esta iteración se trabajará en la funcionalidad de tirar los dados, y por tanto, mover a un personaje a la habitación deseada.

  \begin{itemize}
    \item Fecha de comienzo: 16/7/2018
    \item Fecha de finalización: 23/7/2018
  \end{itemize}

  \item \textbf{Sexta Iteración}: En esta iteración se trabajará en la funcionalidad de realizar una anotación.

  \begin{itemize}
    \item Fecha de comienzo: 23/7/2018
    \item Fecha de finalización: 30/7/2018
  \end{itemize}

  \item \textbf{Séptima Iteración}: En esta iteración se trabajará en la funcionalidad de realizar una acusación y por tanto en la pantalla de ganar el juego.

  \begin{itemize}
    \item Fecha de comienzo: 30/7/2018
    \item Fecha de finalización: 6/8/2018
  \end{itemize}
\end{itemize}

\hfill

\textbf{Entrega 4}: Esta entrega consistirá en la aplicación a la que se habrá añadido la funcionalidad de pasar de turno al del siguiente jugador, y por tanto mostrar la información correspondiente al otro jugador.

\begin{itemize}
  \item Fecha de entrega: 27/8/2018
\end{itemize}

\hfill

Este entrega se compone de las siguientes iteraciones:

\begin{itemize}
  \item \textbf{Octava Iteración}:  En esta iteración se trabajará en la funcionalidad de que el jugador pueda pasar de turno, y por tanto, en el turno del otro jugador se muestre información diferente.

  \begin{itemize}
    \item Fecha de comienzo: 6/8/2018
    \item Fecha de finalización: 27/8/2018
  \end{itemize}
\end{itemize}

%%%%%%%%%%%%%%%%%%%%%%%%%%%%%%%%%%%%%%%%%%%%% HISTORIAS DE USUARIO %%%%%%%%%%%%%%%%%%%%%%%%%%%%%%%%%%%%%%%%%%%%%%%%%%%%%%%%%%

\section{Historias de usuario}
Se han llevado a cabo historias de usuario, para poniendonos en lugar del usuario, crear diferentes situaciones con las que el usuario se encontrará mientras utilice el juego, y de esta forma, saber cuales son los requisitos que nuestro software tiene que cumplir.\\

A partir de esta tabla de historias de usuario, se ha creado el Product Backlog que se puede encontrar en el Apéndice \ref{apendice-product-backlog}, en este se indican las iteraciones y entregas en las que se situa cada historia de usuario.\\

También se ha realizado de forma informal un Sprint Backlog para cada iteración con las historias de usuario a completar en dicha iteración, no obstante estos no se han incluido en este documento, ya que al ser un equipo de desarrollo de una persona no son relevantes.\\

En la Tabla \ref{tabla-lista-historias-usuario} podemos encontrar todas las historias de usuario.

%%%%%%%%%%%%%%%%%%%%%%%%%%%%%%%%%%%%%%%%%%%%%%%% TABLA LISTA HISTORIAS USUARIO %%%%%%%%%%%%%%%%%%%%%%%%%%%%%%%%%%%%%%%%%%%%%%%%%%%

\begin{table}[h]
  \begin{center}
    \begin{tabular}{|p{1cm}|p{7.5cm}|p{1.9cm}|p{1.6cm}|}

      \hline
        \rowcolor{Gray} \textbf{ID}
        & \textbf{Título}
        & \textbf{Estimación}
        & \textbf{Prioridad}\\

      \hline
      HU1
      & Seleccionar personajes
      & 2
      & 1\\

      \hline
      HU2
      & Comenzar juego
      & 3
      & 1\\

      \hline
      HU3
      & Salir de la pantalla de instrucciones
      & 4
      & 1\\

      \hline
      HU4
      & Escanear tablero
      & 5
      & 1\\

      \hline
      HU5
      & Tirar los dados
      & 8
      & 1\\

      \hline
      HU6
      & Seleccionar habitación a la que desplazarse
      & 2
      & 1\\

      \hline
      HU7
      & Acceder al menú de anotaciones
      & 2
      & 1\\

      \hline
      HU8
      & Marcar personaje con una interrogación
      & 2
      & 1\\

      \hline
      HU9
      & Marcar arma con una interrogación
      & 1
      & 1\\

      \hline
      HU10
      & Marcar habitación con una interrogación
      & 1
      & 1\\

      \hline
      HU11
      & Marcar personaje con una X
      & 2
      & 1\\

      \hline
      HU12
      & Marcar arma con una X
      & 1
      & 1\\

      \hline
      HU13
      & Marcar habitación con una X
      & 1
      & 1\\

      \hline
      HU14
      & Escanear una acusación
      & 8
      & 1\\

      \hline
      HU15
      & Terminar partida
      & 2
      & 1\\

      \hline
      HU16
      & Cambiar de turno
      & 1
      & 1\\

      \hline

    \end{tabular}

    \caption{Listado de historias de usuario.}
    \label{tabla-lista-historias-usuario}

  \end{center}
\end{table}

\newpage

Las historias de usuario se componen de un identificador y título, de una descripción que contiene la acción que el usuario quiere realizar, la estimación de tiempo que llevará realizara, en que entrega hay que realizarla y que pruebas de aceptación hay que pasarle para comprobar su correcto funcionamiento. La siguiente tabla se corresponde con la primera historia de usuario desarrollada para el juego:

\begin{table}[h]
  \begin{center}
    \begin{tabular}{|p{4cm}|p{4cm}|p{4cm}|}

    \hline
    \textbf{Identificador}: HU.1
    & \multicolumn{2}{p{8cm}|}{Seleccionar personajes}\\

    \hline
    \multicolumn{3}{|p{12cm}|}{\textbf{Descripción}: Como usuario jugador, quiero poder seleccionar un personaje de los disponibles en el juego.}\\

    \hline
    \textbf{Estimación}:2
    & \textbf{Prioridad}: 1
    & \textbf{Entrega}: 2\\

    \hline
    \multicolumn{3}{|p{12cm}|}{\textbf{Pruebas de aceptación}:
      \begin{itemize}
        \item Comprobar que los personajes elegidos se almacena correctamente.
      \end{itemize}
    }\\

    \hline

    \end{tabular}

    \caption{Tabla de la historia de usuario número 1.}
    \label{tabla-hu1}

  \end{center}
\end{table}

En el Apéndice \ref{historias-usuario} podemos encontrar las historias de usuario con detalle, desde la Tabla \ref{tabla-hu1} hasta la Tabla \ref{tabla-hu16}.

\chapter{Plan de entregas}
\label{ch:plan}

%%%%%%%%%%%%%%%%%%%%%%%%%%%%%%%%%%%%%%%%%%%%%%%%%% PLAN DE ENTREGAS %%%%%%%%%%%%%%%%%%%%%%%%%%%%%%%%%%%%%%%%%%%%%%%%%%%%%%

\section{Plan de entregas}

\textbf{Entrega 0}: Esta entrega consistirá en una evaluación de la viabilidad del proyecto en la que se desarrollará una aplicación que sea capaz de utilizando ARCore mostrar diferentes objetos 3D asociados a diferentes imágenes, cuando las escanea.

\begin{itemize}
  \item Fecha de entrega: 26/6/2018
\end{itemize}

\textbf{Entrega 1}: Esta entrega consistirá en bocetos a papel del juego, abarcando toda la funcionalidad de éste, y un informe que contendrá la información acerca de pruebas con diferentes usuarios sobre la usabilidad de la aplicación.

\begin{itemize}
  \item Fecha de entrega: 2/7/2018
\end{itemize}

\textbf{Entrega 2}: Esta entrega consistirá la aplicación que tendrá desarrollada la pantalla inicial de la aplicación, la pantalla de instrucciones y la pantalla de juego, en la que a partir del tablero se mostrarán los elementos (con realidad aumentada) necesarios para el juego.

\begin{itemize}
  \item Fecha de entrega: 16/7/2018
\end{itemize}

\textbf{Entrega 3}: Esta entrega consistirá en la aplicación a la que se habrá añadido la funcionalidad de tirar los dados, y por tanto mover al personaje a la habitación deseada, la funcionalidad de realizar una anotación, y la funcionalidad de hacer una acusación.

\begin{itemize}
  \item Fecha de entrega: 6/8/2018
\end{itemize}

\textbf{Entrega 4}: Esta entrega consistirá en la aplicación a la que se habrá añadido la funcionalidad de pasar de turno al del siguiente jugador, y por tanto mostrar la información correspondiente al otro jugador.

\begin{itemize}
  \item Fecha de entrega: 27/8/2018
\end{itemize}

\section{Historias de usuario}
En la Tabla \ref{tabla-lista-historias-usuario} podemos encontrar todas las historias de usuario.

%%%%%%%%%%%%%%%%%%%%%%%%%%%%%%%%%%%%%%%%%%%%%%%% TABLA LISTA HISTORIAS USUARIO %%%%%%%%%%%%%%%%%%%%%%%%%%%%%%%%%%%%%%%%%%%%%%%%%%%

\begin{table}[h]
  \begin{center}
    \begin{tabular}{|p{1cm}|p{7.5cm}|p{1.9cm}|p{1.6cm}|}

      \hline
        \rowcolor{Gray} \textbf{ID}
        & \textbf{Título}
        & \textbf{Estimación}
        & \textbf{Prioridad}\\

      \hline
      HU1
      & Seleccionar personajes
      & 2
      & 1\\

      \hline
      HU2
      & Comenzar juego
      & 3
      & 1\\

      \hline
      HU3
      & Salir de la pantalla de instrucciones
      & 4
      & 1\\

      \hline
      HU4
      & Escanear tablero
      & 5
      & 1\\

      \hline
      HU5
      & Tirar los dados
      & 8
      & 1\\

      \hline
      HU6
      & Seleccionar habitación a la que desplazarse
      & 2
      & 1\\

      \hline
      HU7
      & Acceder al menú de anotaciones
      & 2
      & 1\\

      \hline
      HU8
      & Marcar personaje con una interrogación
      & 2
      & 1\\

      \hline
      HU9
      & Marcar arma con una interrogación
      & 1
      & 1\\

      \hline
      HU10
      & Marcar habitación con una interrogación
      & 1
      & 1\\

      \hline
      HU11
      & Marcar personaje con una X
      & 2
      & 1\\

      \hline
      HU12
      & Marcar arma con una X
      & 1
      & 1\\

      \hline
      HU13
      & Marcar habitación con una X
      & 1
      & 1\\

      \hline
      HU14
      & Escanear una acusación
      & 8
      & 1\\

      \hline
      HU15
      & Terminar partida
      & 2
      & 1\\

      \hline
      HU16
      & Cambiar de turno
      & 1
      & 1\\

      \hline

    \end{tabular}

    \caption{Listado de historias de usuario.}
    \label{tabla-lista-historias-usuario}

  \end{center}
\end{table}

En la Tabla \ref{tabla-product-backlog} podemos encontrar el Product Backlog.
%%%%%%%%%%%%%%%%%%%%%%%%%%%%%%%%%%%%%%%%%%%%%%%%%%TABLA PRODUCT BACKLOG%%%%%%%%%%%%%%%%%%%%%%%%%%%%%%%%%%%%%%%%%%%%%%%%%%%%%%

\begin{table}[h]
  \begin{center}
    \begin{tabular}{|p{1cm}|p{7.5cm}|p{1.9cm}|p{1.6cm}|p{1.6cm}|}

      \hline
        \rowcolor{Gray} \textbf{ID}
        & \textbf{Título}
        & \textbf{Estimación}
        & \textbf{Iteración}
        & \textbf{Entrega}\\

      \hline
      HU1
      & Seleccionar personajes
      & 2
      & 3
      & 2\\

      \hline
      HU2
      & Comenzar juego
      & 3
      & 3
      & 2\\

      \hline
      HU3
      & Salir de la pantalla de instrucciones
      & 4
      & 3
      & 2\\

      \hline
      HU4
      & Escanear tablero
      & 5
      & 4
      & 2\\

      \hline
      HU5
      & Tirar los dados
      & 8
      & 5
      & 3\\

      \hline
      HU6
      & Seleccionar habitación a la que desplazarse
      & 2
      & 5
      & 3\\

      \hline
      HU7
      & Acceder al menú de anotaciones
      & 2
      & 6
      & 3\\

      \hline
      HU8
      & Marcar personaje con una interrogación
      & 2
      & 6
      & 3\\

      \hline
      HU9
      & Marcar arma con una interrogación
      & 1
      & 6
      & 3\\

      \hline
      HU10
      & Marcar habitación con una interrogación
      & 1
      & 6
      & 3\\

      \hline
      HU11
      & Marcar personaje con una X
      & 2
      & 6
      & 3\\

      \hline
      HU12
      & Marcar arma con una X
      & 1
      & 6
      & 3\\

      \hline
      HU13
      & Marcar habitación con una X
      & 1
      & 6
      & 3\\

      \hline
      HU14
      & Escanear una acusación
      & 8
      & 7
      & 3\\

      \hline
      HU15
      & Terminar partida
      & 2
      & 4
      & 2\\

      \hline
      HU16
      & Cambiar de turno
      & 1
      & 8
      & 4\\

      \hline

    \end{tabular}

    \caption{Listado del Product Backlog.}
    \label{tabla-product-backlog}

  \end{center}
\end{table}

En el Apéndice \ref{historias-usuario} podemos encontrar las historias de usuario con detalle, desde la Tabla \ref{tabla-hu1} hasta la Tabla \ref{tabla-hu16}.

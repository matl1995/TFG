\addcontentsline{toc}{chapter}{Bibliografía}

\subsection*{Páginas web consultadas durante la realización del proyecto}

\bibitem{intel}
{\tt Intel: Demystifying the Virtual Reality Landscape. Recuperado de }\\
\url{https://www.intel.es/content/www/es/es/tech-tips-and-tricks/virtual-reality-vs-augmented-reality.html}

\bibitem{arcore}
{\tt Google Developers: ARCore Overview. Recuperado de }\\
\url{https://developers.google.com/ar/discover/}

\bibitem{arkit}
{\tt Apple Developer: ARKit. Recuperado de }\\
\url{https://developer.apple.com/arkit/}

\bibitem{vuforia}
{\tt Vuforia: AR Features. Recuperado de }\\
\url{https://www.vuforia.com/features.html}

\bibitem{easyar}
{\tt EasyAR: EasyAR SDK. Recuperado de }\\
\url{https://www.easyar.com/view/sdk.html}

\bibitem{wikitude}
{\tt Wikitude: Augmented Reality SDK. Recuperado de }\\
\url{https://www.wikitude.com/products/wikitude-sdk/}

\bibitem{windows-mixed-reality}
{\tt Windows Mixed Reality: Inmerse yourself in a new reality. Recuperado de }\\
\url{https://www.microsoft.com/en-us/windows/windows-mixed-reality}

\bibitem{ndk}
{\tt Android Developers: Cómo comenzar a usar el NDK. Recuperado de }\\
\url{https://developer.android.com/ndk/guides/}

\bibitem{unity-scripting}
{\tt Unity - Manual: Creando y usando scripts. Recuperado de }\\
\url{https://docs.unity3d.com/es/current/Manual/CreatingAndUsingScripts.html}

\bibitem{unreal}
{\tt Unreal Engine: Programming Quick Start. Recuperado de }\\
\url{https://docs.unrealengine.com/en-US/Programming/QuickStart}

\bibitem{arcore-web}
{\tt Android Developers: Quickstart for AR on the Web. Recuperado de }\\
\url{https://developers.google.com/ar/develop/web/quickstart}

\bibitem{arcore-augmented-images}
{\tt Android Developers: Recognize and Augment Images. Recuperado de }\\
\url{https://developers.google.com/ar/develop/java/augmented-images/}

\bibitem{arcore-anchors}
{\tt Android Developers: Working with Anchors. Recuperado de }\\
\url{https://developers.google.com/ar/develop/developer-guides/anchors}

\bibitem{arcore-cloud-anchors}
{\tt Android Developers: Share AR Experiences with Cloud Anchors. Recuperado de }\\
\url{https://developers.google.com/ar/develop/java/cloud-anchors/cloud-anchors-overview-android}

\bibitem{arcore-sceneform}
{\tt Anroid Developers: Sceneform Overview. Recuperado de }\\
\url{https://developers.google.com/ar/develop/java/sceneform/}

\bibitem{reinoso}
{\tt Realidad Aumentada y Virtual: Descubriendo sus posibilidades en Educación y Formación 1/2 - #EDUinsights. [Diapositivas de Power Point]. Recuperado de }\\
\url{https://es.slideshare.net/tecnotic/realidad-aumentada-y-virtual-descubriendo-sus-posibilidades-en-educacin-y-formacin-12-eduinsights-91588341?qid=96d26ead-6ee5-4f87-9c8c-846931c34eee&v=&b=&from_search=1  } . {\tt Reinoso, R. (22 de marzo de 2018).}

\bibitem{ar-studio}
{\tt Facebook for Developers: AR Studio - Documentation. Recuperado de }\\
\url{https://developers.facebook.com/docs/ar-studio}

\bibitem{layar}
{\tt Layar: Easily create your own interactive augmented reality experiences. Recuperado de }\\
\url{https://www.layar.com/}

\bibitem{mira-ar}
{\tt Mira Augmented Reality: Deploy Augmented Reality. Recuperado de }\\
\url{https://www.mirareality.com/}

\bibitem{magic-leap}
{\tt Magic Leap: Magic in the Making. Recuperado de }\\
\url{https://www.magicleap.com/}

\bibitem{meta-vision}
{\tt Meta: Powered by neuroscience. Designed for trailblazers. Recuperado de }\\
\url{http://www.metavision.com/}

\bibitem{hololens}
{\tt Microsoft: Microsoft HoloLens. Recuperado de }\\
\url{https://www.microsoft.com/es-es/hololens}


\subsection*{Libros consultados}

\bibitem{azuma}
{\em A survey of augmented reality. Presence: Teleoperators & Virtual Environments}, 6(4), 355-385. Azuma, R. T. (1997)

\bibitem{ullmer}
{\em Emerging frameworks for tangible user interfaces. IBM systems journal}, 39(3.4), 915-931. Ullmer, B., & Ishii, H. (2000).

\bibitem{ha}
{\em ARtalet: tangible user interface based immersive augmented reality authoring tool for Digilog book. In Ubiquitous Virtual Reality (ISUVR)}, 2010 International Symposium on (pp. 40-43). IEEE. Ha, T., Woo, W., Lee, Y., Lee, J., Ryu, J., Choi, H., & Lee, K. (2010, July).

\bibitem{van-krevelen}
{\em A survey of augmented reality technologies, applications and limitations. International journal of virtual reality},  9(2), 1. Van Krevelen, D. W. F., & Poelman, R. (2010).

\bibitem{prendes-espinosa}
{\em Realidad aumentada y educación: análisis de experiencias prácticas. Pixel-Bit. Revista de Medios y Educación}, 46, 187-203. Prendes Espinosa, C. (2015).

\bibitem{billinghurst}
{\em A survey of augmented reality. Foundations and Trends® in Human–Computer Interaction}, 8(2-3), 73-272. Billinghurst, M., Clark, A., & Lee, G. (2015).

\bibitem{likamwa}
{\em Draining our glass: An energy and heat characterization of google glass. In Proceedings of 5th Asia-Pacific Workshop on Systems (p. 10). ACM.} LiKamWa, R., Wang, Z., Carroll, A., Lin, F. X., & Zhong, L. (2014, June).

\bibitem{ine}
{\em Encuesta sobre Equipamiento y Uso de Tecnologías de la Información y Comunicación en los Hogares.} INE, B. (2017).


\subsection*{Páginas de principales productos y programas usados}

\bibitem{unity} {\tt Unity - Página principal}\\
\url{https://unity3d.com/es}

\subsection*{Otros recursos utilizados}

Apuntes de asignaturas del \textbf{Grado en Ingeniería Informática}, tales como:
\begin{itemize}
    \item Programación de Dispositivos Móviles
    \item Fundamentos de Ingeniería del Software
    \item Dirección y Gestión de proyectos
    \item Metodologías de Desarrollo Ágil
\end{itemize}

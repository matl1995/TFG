\addcontentsline{toc}{chapter}{Bibliografía}

\bibitem{ine}
INE, B., {\em Encuesta sobre Equipamiento y Uso de Tecnologías de la Información y Comunicación en los Hogares.}, Notas de prensa del Instituto Nacional de Estadística, (5 de octubre de 2017).

\bibitem{azuma}
Azuma, R. T., {\em A survey of augmented reality. Presence: Teleoperators & Virtual Environments}, 6(4), 355-385. (1997).

\bibitem{intel}
Intel, {\em Demystifying the Virtual Reality Landscape}, Virtual Reality Vs. Augmented Reality Vs. Mixed Reality, Recuperado de \\
\url{https://www.intel.es/content/www/es/es/tech-tips-and-tricks/virtual-reality-vs-augmented-reality.html}, (2018).

\bibitem{reinoso}
Reinoso, R., {\em Realidad Aumentada y Virtual: Descubriendo sus posibilidades en Educación y Formación 1/2 - #EDUinsights.}, Slideshare, [Diapositivas de Power Point]. Recuperado de \\
\url{https://es.slideshare.net/tecnotic/realidad-aumentada-y-virtual-descubriendo-sus-posibilidades-en-educacin-y-formacin-12-eduinsights-91588341?qid=96d26ead-6ee5-4f87-9c8c-846931c34eee&v=&b=&from_search=1  }, (22 de marzo de 2018).

\bibitem{van-krevelen}
Van Krevelen, D. W. F., & Poelman, R., {\em A survey of augmented reality technologies, applications and limitations. International journal of virtual reality},  9(2), 1. (2010).

\bibitem{prendes-espinosa}
Prendes Espinosa, C., {\em Realidad aumentada y educación: análisis de experiencias prácticas. Pixel-Bit. Revista de Medios y Educación}, 46, 187-203. (2015).

\bibitem{billinghurst}
Billinghurst, M., Clark, A., & Lee, G., {\em A survey of augmented reality. Foundations and Trends® in Human–Computer Interaction}, 8(2-3), 73-272. (2015).

\bibitem{hololens}
Microsoft, Microsoft HoloLens, Microsoft, Recuperado de \\
\url{https://www.microsoft.com/es-es/hololens}, (2018).

\bibitem{likamwa}
LiKamWa, R., Wang, Z., Carroll, A., Lin, F. X., & Zhong, L., {\em Draining our glass: An energy and heat characterization of google glass. In Proceedings of 5th Asia-Pacific Workshop on Systems (p. 10). ACM.} (Junio de 2014).

\bibitem{meta-vision}
Meta Company, {\em Powered by neuroscience. Designed for trailblazers.}, Meta, Recuperado de \\
\url{http://www.metavision.com/}, (2017).

\bibitem{magic-leap}
Magic Leap, {\em Magic in the Making}, Magic Leap, Recuperado de \\
\url{https://www.magicleap.com/}, (2018).

\bibitem{mira-ar}
Mira Labs, {\em Deploy Augmented Reality}, Mira Augmented Reality, Recuperado de \\
\url{https://www.mirareality.com/}, (2018).

\bibitem{arkit}
Apple, {\em ARKit}, Apple Developer, Recuperado de \\
\url{https://developer.apple.com/arkit/}, (2018).

\bibitem{vuforia}
Vuforia, {\em AR Features}, Vuforia Engine, Recuperado de \\
\url{https://www.vuforia.com/features.html}, (2018).

\bibitem{easyar}
EasyAR, {\em What is EasyAR SDK}, EasyAR, Recuperado de \\
\url{https://www.easyar.com/view/sdk.html}, (2018).

\bibitem{wikitude}
Wikitude, {\em The world’s leading Augmented Reality SDK}, Wikitude, Recuperado de \\
\url{https://www.wikitude.com/products/wikitude-sdk/}, (2018).

\bibitem{ar-studio}
Facebook, {\em AR Studio}, Facebook for developers, Recuperado de \\
\url{https://developers.facebook.com/docs/ar-studio}, (2018).

\bibitem{layar}
Layar, {\em Easily create your own interactive augmented reality experiences}, Layar, Recuperado de }\\
\url{https://www.layar.com/}, (2018).

\bibitem{arcore}
Google, {\em ARCore Overview}, Google Developers, Recuperado de \\
\url{https://developers.google.com/ar/discover/}, (8 de mayo de 2018).

\bibitem{arcore-augmented-images}
Google, {\em Recognize and Augment Images}, Android Developers, Recuperado de \\
\url{https://developers.google.com/ar/develop/java/augmented-images/}, (11 de mayo de 2018).

\bibitem{arcore-anchors}
Google, {\em Working with Anchors}, Android Developers, Recuperado de \\
\url{https://developers.google.com/ar/develop/developer-guides/anchors}, (8 de mayo de 2018).

\bibitem{arcore-cloud-anchors}
Google, {\em Share AR Experiences with Cloud Anchors}, Android Developers, Recuperado de \\
\url{https://developers.google.com/ar/develop/java/cloud-anchors/cloud-anchors-overview-android}, (8 de mayo de 2018).

\bibitem{arcore-sceneform}
Google, {\em Sceneform Overview}, Android Developers, Recuperado de \\
\url{https://developers.google.com/ar/develop/java/sceneform/}, (25 de junio de 2018).

\bibitem{ndk}
Google, {\em Cómo comenzar a usar el NDK}, Android Developers, Recuperado de \\
\url{https://developer.android.com/ndk/guides/}, (19 de abril de 2018).

\bibitem{unity-scripting}
Unity, {\em Creando y usando scripts}, Unity-Manual, Recuperado de \\
\url{https://docs.unity3d.com/es/current/Manual/CreatingAndUsingScripts.html}, (2016).

\bibitem{unreal}
Epic Games, {\em Programming Quick Start}, Unreal Engine, Recuperado de \\
\url{https://docs.unrealengine.com/en-US/Programming/QuickStart}, (2018).

\bibitem{arcore-web}
Google, {\em Quickstart for AR on the Web}, Android Developers, Recuperado de \\
\url{https://developers.google.com/ar/develop/web/quickstart}, (23 de febrero de 2018).

\bibitem{ullmer}
Ullmer, B., & Ishii, H., {\em Emerging frameworks for tangible user interfaces. IBM systems journal}, 39(3.4), 915-931. (2000).

\bibitem{windows-mixed-reality}
Microsoft, {\em Inmerse yourself in a new reality}, Windows Mixed Reality, Recuperado de \\
\url{https://www.microsoft.com/en-us/windows/windows-mixed-reality}, (2018).

\bibitem{ha}
Ha, T., Woo, W., Lee, Y., Lee, J., Ryu, J., Choi, H., & Lee, K., {\em ARtalet: tangible user interface based immersive augmented reality authoring tool for Digilog book. In Ubiquitous Virtual Reality (ISUVR)}, 2010 International Symposium on (pp. 40-43). IEEE. (Julio de 2010).

\bibitem{beck}
Beck, K., Beedle, M., Van Bennekum, A., Cockburn, A., Cunningham, W., Fowler, M., ... & Kern, J., {\em Manifesto for agile software development}. (2001).

\bibitem{sanchez}
Sánchez, J., {\em En busca del Diseño Centrado en el Usuario (DCU): definiciones, técnicas y una propuesta}, No solo usabilidad: Revista sobre personas, diseño y tecnología, Recuperado de \\
\url{http://www.nosolousabilidad.com/articulos/dcu.htm?utm_source=iNeZha.com&utm_medium=im_robot&utm_campaign=iNezha}, (5 de septiembre de 2011).

\bibitem{pereira}
Pereira, A. M. M., {\em El proceso productivo del videojuego: fases de producción/The production process of the game: production phases}. Historia y Comunicación Social, 19, 791-805. (2014).

\bibitem{nielsen}
Nielsen, J., {\em 10 usability heuristics for user interface design}. Nielsen Norman Group, 1(1). (1995).
